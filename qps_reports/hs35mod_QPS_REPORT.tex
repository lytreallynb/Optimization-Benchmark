\documentclass[a4paper,11pt]{article}
\usepackage[UTF8]{ctex}
\usepackage{amsmath, amssymb, booktabs, geometry, longtable, xcolor, fancyhdr, array}
\usepackage{breqn}  % 用于自动换行的数学公式
\geometry{a4paper, left=1.5cm, right=1.5cm, top=2.5cm, bottom=2.5cm}

% 允许公式跨页
\allowdisplaybreaks[4]

% 数学字体设置
\usepackage{mathptmx}  % 使用Times字体
\renewcommand{\familydefault}{\rmdefault}

% 适中的行距设置
\usepackage{setspace}
\setstretch{1.1}

% 数学公式的间距设置
\setlength{\abovedisplayskip}{6pt}
\setlength{\belowdisplayskip}{6pt}
\setlength{\abovedisplayshortskip}{3pt}
\setlength{\belowdisplayshortskip}{3pt}

\title{QPS格式二次规划问题分析报告\\{\large hs35mod}}
\author{QPS解析器}
\date{2025年07月14日 11:31:07}

\pagestyle{fancy}
\fancyhf{}
\fancyhead[L]{QPS分析报告}
\fancyhead[R]{hs35mod}
\fancyfoot[C]{\thepage}

\begin{document}
\maketitle
\tableofcontents
\newpage

\section{问题概述}
\subsection{基本信息}
\begin{table}[h!]
\centering
\begin{tabular}{ll}
\toprule
\textbf{属性} & \textbf{值} \\
\midrule
问题名称 & \texttt{} \\
源文件 & \texttt{hs35mod.qps} \\
问题类型 & 二次规划 (QP) \\
变量数量 & 3 \\
约束数量 & 1 \\
二次项数 & 7 \\
目标常数 & 9.0 \\
\bottomrule
\end{tabular}
\caption{问题基本信息}
\end{table}

\section{数学模型}
\subsection{优化模型}

\textbf{目标函数:}

\begin{align}
\min\quad f(x) &= 9 - 8\,C_{1} - 6\,C_{2} - 4\,C_{3} \nonumber\\
&\quad + \frac{1}{2} \sum_{i,j} Q_{ij} x_i x_j\label{eq:objective}
\end{align}

\textbf{二次项矩阵特征:}
\begin{itemize}
\item 对角线项: 3 个
\item 非对角线项: 2 个
\item 矩阵类型: 对称正定
\end{itemize}

\textbf{约束条件:}
\begin{align}
-1\,C_{1} - 1\,C_{2} - 2\,C_{3} &\geq -3 \nonumber
\end{align}

\textbf{变量边界:}
\begin{itemize}
\item 非负变量 $x_i \geq 0$: 2 个变量
\item 下界变量 $x_i \geq 0.5$: 1 个变量
\end{itemize}

\section{求解结果}

\subsection{最优解}
\begin{itemize}
\item \textbf{求解状态:} \textcolor{green}{最优解}
\item \textbf{目标函数值:} $0.25$
\item \textbf{求解时间:} 0.009 秒
\item \textbf{非零变量:} 3/3
\end{itemize}

\subsection{解的分析}
\begin{itemize}
\item 非零变量平均值: 0.833333
\item 非零变量标准差: 0.471405
\item 最大变量值: 1.5
\item 最小变量值: 0.499999
\end{itemize}


\subsection{所有非零变量值}
\begin{longtable}{p{2.5cm}@{\hspace{0.5em}}r@{\hspace{0.8em}}p{3.5cm}}
\toprule
\textbf{变量} & \textbf{值} & \textbf{边界} \\
\midrule
\endfirsthead
\multicolumn{3}{c}{\textit{续表}} \\
\toprule
\textbf{变量} & \textbf{值} & \textbf{边界} \\
\midrule
\endhead
\bottomrule
\endfoot
\bottomrule
\endlastfoot
$C_{1}$ & 1.500000 & [0, ∞) \\
$C_{2}$ & 0.500000 & [0.5, 0.5] \\
$C_{3}$ & 0.499999 & [0, ∞) \\
\bottomrule
\caption{所有非零变量值(按绝对值排序)}
\end{longtable}
\end{document}