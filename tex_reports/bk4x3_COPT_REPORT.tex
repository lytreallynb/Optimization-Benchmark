
\documentclass[a4paper,10pt]{article}
\usepackage[UTF8]{ctex}
\usepackage{amsmath}
\usepackage{longtable}
\usepackage{booktabs}
\usepackage{geometry}
\geometry{a4paper, left=1.5cm, right=1.5cm, top=2cm, bottom=2cm}
\usepackage{fancyhdr}
\pagestyle{fancy}
\fancyhf{}
\fancyhead[C]{MPS文件数学优化模型求解报告}
\fancyfoot[C]{\thepage}

% 允许数学环境跨页
\allowdisplaybreaks[4]
% 减小数学环境的间距
\setlength{\abovedisplayskip}{6pt}
\setlength{\belowdisplayskip}{6pt}
\setlength{\abovedisplayshortskip}{3pt}
\setlength{\belowdisplayshortskip}{3pt}

\title{数学优化模型求解报告\\{\large bk4x3.mps}}
\author{COPT求解器}
\date{求解时间: 2025年07月09日 22:15:23}

\begin{document}
\maketitle
\tableofcontents
\newpage

\section{模型概览}

\textbf{文件名:} \texttt{bk4x3.mps} \\
\textbf{模型名:} 未知模型 \\
\textbf{变量总数:} 24 \\
\textbf{约束总数:} 19 \\
\textbf{求解时间:} 2025年07月09日 22:15:23 \\
\textbf{优化方向:} Minimize \\

\textbf{模型类型:} 混合整数规划 (MIP)

\section{目标函数}

\textbf{目标函数摘要:}
\begin{equation}
\min \quad Z = \sum_{i} c_i Y_i + \sum_{j} d_j X_j
\end{equation}

Y变量: 12个,系数范围 [10, 30] \\
X变量: 10个,系数范围 [2, 5]

\textbf{完整目标函数:}

\allowdisplaybreaks
{\small
\begin{align}
\min \quad Z = &\; 2 X_{0} + 3 X_{1} + 4 X_{2} \\[0.3ex]
&\;  + 3 X_{3} + 2 X_{4} + 4 X_{7} \\[0.3ex]
&\;  + 3 X_{8} + 4 X_{9} + 5 X_{10} \\[0.3ex]
&\;  + 2 X_{11} + 10 Y_{0} + 30 Y_{1} \\[0.3ex]
&\;  + 20 Y_{2} + 10 Y_{3} + 30 Y_{4} \\[0.3ex]
&\;  + 20 Y_{5} + 10 Y_{6} + 30 Y_{7} \\[0.3ex]
&\;  + 20 Y_{8} + 10 Y_{9} + 30 Y_{10} \\[0.3ex]
&\;  + 20 Y_{11}\nonumber
\end{align}
}

\section{约束条件}

\subsection{等式约束 (7个)}

\allowdisplaybreaks
{\small
\begin{align}
X_{0} + X_{1} + X_{2} &= 10 && \text{(A0)} \\
X_{3} + X_{4} + X_{5} &= 30 && \text{(A1)} \\
X_{6} + X_{7} + X_{8} &= 40 && \text{(A2)} \\
X_{9} + X_{10} + X_{11} &= 20 && \text{(A3)} \\
X_{0} + X_{3} + X_{6} + X_{9} &= 20 && \text{(B0)} \\
\allowbreak
X_{1} + X_{4} + X_{7} + X_{10} &= 50 && \text{(B1)} \\
X_{2} + X_{5} + X_{8} + X_{11} &= 30 && \text{(B2)} \\
\end{align}
}

\subsection{不等式约束 (14个)}

\allowdisplaybreaks
{\small
\begin{align}
X_{0} - 10Y_{0} &\leq 0 && \text{(G0)} \\
X_{1} - 10Y_{1} &\leq 0 && \text{(G1)} \\
X_{2} - 10Y_{2} &\leq 0 && \text{(G2)} \\
X_{3} - 20Y_{3} &\leq 0 && \text{(G3)} \\
X_{4} - 30Y_{4} &\leq 0 && \text{(G4)} \\
X_{5} - 30Y_{5} &\leq 0 && \text{(G5)} \\
X_{6} - 20Y_{6} &\leq 0 && \text{(G6)} \\
X_{7} - 40Y_{7} &\leq 0 && \text{(G7)} \\
X_{8} - 30Y_{8} &\leq 0 && \text{(G8)} \\
X_{9} - 20Y_{9} &\leq 0 && \text{(G9)} \\
\allowbreak
X_{10} - 20Y_{10} &\leq 0 && \text{(G10)} \\
X_{11} - 20Y_{11} &\leq 0 && \text{(G11)} \\
\end{align}
}

\section{变量定义}

\subsection{二元变量 (12个)}

\begin{equation}
Y_i \in \{0,1\}, \quad i \in \{0, 1, 2, \ldots, 11\}
\end{equation}

\textbf{所有二元变量:}

{\small
$Y_{0}$, $Y_{1}$, $Y_{2}$, $Y_{3}$, $Y_{4}$, $Y_{5}$, $Y_{6}$, $Y_{7}$, $Y_{8}$, $Y_{9}$, \\
$Y_{10}$, $Y_{11}$
}

\subsection{连续变量 (12个)}

所有连续变量均为非负实数:
\begin{equation}
X_j \geq 0, \quad j \in \{0, 1, 2, \ldots, 11\}
\end{equation}

\textbf{连续变量说明:} 模型包含12个连续决策变量,所有变量的取值范围均为非负实数域。

\section{求解结果}

\subsection{最优目标值}

最优目标值为: $\mathbf{350}$

\subsection{求解状态}

求解状态: \textbf{最优解}

\subsection{最优解}

各决策变量的最优取值如下:

\begin{center}
\begin{longtable}{cc}
\toprule
\textbf{变量名} & \textbf{最优值} \\
\midrule
\endfirsthead
\multicolumn{2}{c}{\textit{续表}} \\
\toprule
\textbf{变量名} & \textbf{最优值} \\
\midrule
\endhead
\bottomrule
\endfoot
\bottomrule
\endlastfoot
\multicolumn{2}{c}{\textbf{二元变量}} \\
\midrule
$Y_{0}$ & 0 \\
$Y_{1}$ & 0 \\
$Y_{2}$ & 1 \\
$Y_{3}$ & 0 \\
$Y_{4}$ & 1 \\
$Y_{5}$ & 0 \\
$Y_{6}$ & 1 \\
$Y_{7}$ & 1 \\
$Y_{8}$ & 0 \\
$Y_{9}$ & 0 \\
$Y_{10}$ & 0 \\
$Y_{11}$ & 1 \\
\midrule
\multicolumn{2}{c}{\textbf{连续变量}} \\
\midrule
$X_{0}$ & 0 \\
$X_{1}$ & 0 \\
$X_{2}$ & 10 \\
$X_{3}$ & 0 \\
$X_{4}$ & 30 \\
$X_{5}$ & 0 \\
$X_{6}$ & 20 \\
$X_{7}$ & 20 \\
$X_{8}$ & 0 \\
$X_{9}$ & 0 \\
$X_{10}$ & 0 \\
$X_{11}$ & 20 \\
\end{longtable}
\end{center}

\section{求解信息汇总}

\textbf{求解器:} COPT \\
\textbf{求解时间:} 2025年07月09日 22:15:23 \\
\textbf{模型规模:} 24个变量, 19个约束 \\
\textbf{最终状态:} 最优解 \\
\textbf{最优目标值:} 350

\end{document}