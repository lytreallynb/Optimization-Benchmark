\documentclass[a4paper, 12pt]{article}

%================================================================================
% Preamble (导言区)
%================================================================================

%--- Page Layout (页面布局) ---
\usepackage[a4paper, margin=1in]{geometry}
\usepackage{parskip} % 段落间增加垂直间距

%--- Font and Language Support (字体与语言支持) ---
\usepackage{xeCJK}
\usepackage{fontspec}
\setCJKmainfont{Noto Serif CJK SC}
\setmainfont{Times New Roman}

%--- Document Styling (文档样式) ---
\usepackage{amsmath}
\usepackage{graphicx}
\usepackage{enumitem}
\usepackage[
    colorlinks=true,
    linkcolor=black,
    urlcolor=blue,
    citecolor=black
]{hyperref}

% 自定义章节标题样式
\usepackage{titlesec}
\titleformat{\section}{\Large\bfseries}{\thesection}{1em}{}
\titleformat{\subsection}{\large\bfseries}{\thesubsection}{1em}{}
\titlespacing*{\section}{0pt}{3.5ex plus 1ex minus .2ex}{2.3ex plus .2ex}

% 代码高亮
\usepackage{minted} 
\setminted{
    fontsize=\small,
    frame=lines,
    breaklines,
    linenos
}

% 流程图绘制
\usepackage{tikz}
\usetikzlibrary{shapes.geometric, arrows.meta, positioning}


%================================================================================
% Document Body (文档正文)
%================================================================================

\begin{document}

%--- Title Section (标题部分) ---
\title{
    \vspace{-1.5cm}
    \textbf{\Huge 求解器技术架构报告}
    \vspace{0.5cm}
}
\author{yutonglv}
\date{\today}
\maketitle
\hrule
\vspace{1cm}

%--- Executive Summary (执行摘要) ---
\begin{abstract}
    \noindent
    本报告旨在全面解析我们优化求解器软件的底层技术架构、用户体验设计、以及未来的自研战略。其核心是\textbf{客户端SDK}与\textbf{核心求解器引擎}的解耦模型,由\textbf{许可证系统}进行访问控制。报告将深入探讨构建此系统所需的技术栈,为交互设计团队提供明确指引,并为公司决策层提供一份从零开始研发求解器的技术部署路线图。
\end{abstract}
\vspace{1cm}

\tableofcontents
\newpage

%--- Section 1: Core Architecture (核心架构) ---
\section{核心架构概述}
我们的求解器软件采用了行业标准的\textbf{三层分离架构},将用户接口、计算逻辑与授权管理完全解耦。这种设计带来了三大优势:
\begin{itemize}
    \item \textbf{灵活性与易用性:} 用户可以通过熟悉的Python SDK进行快速开发和集成。
    \item \textbf{高性能:} 核心计算任务由独立、高效的C++原生引擎执行,不受脚本语言性能限制。
    \item \textbf{知识产权保护:} 核心算法被编译成二进制黑盒,有效防止了逆向工程。
\end{itemize}

下面的流程图直观地展示了用户调用求解器时的内部工作流:
\begin{figure}[h!]
    \centering
    \begin{tikzpicture}[
        node distance=2.8cm and 2cm,
        process/.style={rectangle, draw, fill=blue!10, text width=8em, text centered, minimum height=3em, rounded corners},
        decision/.style={diamond, draw, fill=orange!20, text width=6em, text centered, aspect=2},
        io/.style={trapezium, trapezium left angle=70, trapezium right angle=110, draw, fill=green!10, minimum height=3em, text centered, text width=8em},
        arrow/.style={-Latex, thick}
    ]
        % Nodes
        \node (user_script) [io] {用户Python脚本 (`model.solve()`)};
        \node (sdk) [process, below=of user_script] {Python SDK};
        \node (license) [decision, below=of sdk] {许可证检查};
        \node (engine) [process, below=of license] {核心求解器引擎 (C++)};
        \node (results) [io, right=of engine, xshift=1cm] {结果文件/输出流};
        
        % Arrows
        \draw [arrow] (user_script) -- node[right, midway, font=\small]{1. 调用} (sdk);
        \draw [arrow] (sdk) -- node[right, midway, font=\small]{2. 请求验证} (license);
        \draw [arrow] (license) -- node[left, midway, font=\small, align=center]{序列化模型数据 \\ (3. 验证通过)} (engine);
        \draw [arrow, red] (license.west) -| ++(-2.5, 0) |- (user_script.west) node[pos=0.25, left, font=\small, align=center]{验证失败, \\ 抛出异常};
        \draw [arrow] (engine) -- node[above, midway, font=\small, align=center]{4. 执行计算, \\ 输出结果} (results);
        \draw [arrow] (results) -| node[below, pos=0.25, font=\small]{5. 解析结果} (sdk);
        \draw [arrow] (sdk) -- node[left, midway, font=\small]{6. 填充模型对象} (user_script);
    \end{tikzpicture}
    \caption{`model.solve()` 调用时的数据流图}
    \label{fig:flowchart}
\end{figure}

%--- Section 2: Technical Stack ---
\section{技术栈深度解析}
要实现上述架构,我们需要组合使用多种技术,构建一个稳健、高效的系统。

\subsection{核心求解器引擎 (Core Solver Engine)}
这是技术壁垒最高的部分,对性能要求极致。
\begin{itemize}
    \item \textbf{编程语言:} \textbf{C++ (17/20)} 是行业标准,它提供了构建高性能、底层内存控制和复杂数据结构所需的一切。
    \item \textbf{构建系统:} \textbf{CMake} 用于管理跨平台(Windows, macOS, Linux)的编译流程。
    \item \textbf{性能库:} 依赖高性能计算库,如 \textbf{Intel MKL} 或开源的 \textbf{OpenBLAS},用于加速底层的矩阵运算。
    \item \textbf{并行计算:} 利用 \textbf{OpenMP} 或 \textbf{Intel TBB} 等技术进行多线程编程,充分利用多核CPU的计算能力。
\end{itemize}

\subsection{Python SDK (软件开发工具包)}
这是直接面向用户的部分,重点在于易用性和封装。
\begin{itemize}
    \item \textbf{核心技术:} 使用 \textbf{Python 3.8+}。
    \item \textbf{C++接口:} 通过 \textbf{ctypes}, \textbf{cffi} 或 \textbf{pybind11} 等库来调用C++核心引擎编译出的动态链接库(.dll/.so/.dylib)。
    \item \textbf{打包与分发:} 使用 \textbf{setuptools} 和 \textbf{wheel} 将SDK打包,并通过 \textbf{PyPI} (Python包索引) 进行发布,实现 `pip install our-solver` 的便捷安装。
\end{itemize}

\subsection{许可证系统 (Licensing System)}
这是商业模式的核心,涉及前后端和加密技术。
\begin{itemize}
    \item \textbf{加密库:} \textbf{OpenSSL} 用于生成非对称密钥对(RSA/ECC),对许可证文件进行签名和验证,防止篡改。
    \item \textbf{许可证服务器后端:} 可使用 \textbf{Python (Django/Flask)} 或 \textbf{Go} 搭建API服务器,处理用户的许可证申请、生成和分发。
    \item \textbf{数据库:} \textbf{PostgreSQL} 或 \textbf{MySQL} 用于存储用户信息、订单和已签发的许可证记录。
\end{itemize}

%--- Section 3: UI/UX Design Guidance ---
\section{给交互设计团队的指导建议}
我们的软件产品不仅仅是代码,其用户体验直接影响客户的购买和使用意愿。以下是给交互设计(UI/UX)团队的设计方向:

\subsection{官方网站与文档门户}
\begin{itemize}
    \item \textbf{首页:} 简洁、专业,有明确的“下载试用”和“购买”入口。突出展示求解器的核心优势和应用案例。
    \item \textbf{文档中心:} 这是用户的生命线。必须做到结构清晰、内容详尽、全文可搜索。提供“快速入门”指南、完整的API参考和丰富的代码示例。
    \item \textbf{个人账户与许可证门户:} 用户登录后,可以方便地查看自己的订单、下载许可证文件、查看机器的Host ID,并管理续费。
\end{itemize}

\subsection{许可证激活工具}
用户在本地与许可证系统交互的唯一工具,设计必须简单明了。
\begin{itemize}
    \item \textbf{形式:} 可以是一个小巧的图形界面(GUI)应用,或一个功能清晰的命令行工具。
    \item \textbf{核心功能:}
        \begin{enumerate}
            \item \textbf{显示Host ID:} 自动检测并清晰展示当前机器的唯一标识码(Host ID),并提供“一键复制”功能。
            \item \textbf{激活许可证:} 提供一个文本框,让用户粘贴从网站获取的许可证密钥。
            \item \textbf{状态反馈:} 激活后,必须给出明确的成功或失败提示。例如:“许可证激活成功,有效期至2026年7月21日。”或“错误:许可证密钥无效或已过期。”
        \end{enumerate}
\end{itemize}

%--- Section 4: In-house Development Roadmap ---
\section{自研求解器技术部署路线图}
如果我们决定从零开始构建自己的商业级优化求解器,这将是一项重大的战略投资,需要分阶段进行。

\subsection{第一阶段:基础研究与算法原型}
\begin{itemize}
    \item \textbf{目标:} 验证核心算法的可行性,并构建一个功能性的原型。
    \item \textbf{团队配置:} 运筹学/应用数学博士 (2-3名), 高性能C++工程师 (1-2名)。
    \item \textbf{关键任务:}
        \begin{enumerate}
            \item 实现线性和混合整数规划的核心算法,如单纯形法、内点法、分支定界法。
            \item 开发内部测试集,并与优秀的开源求解器(如 HiGHS, GLPK)进行精度和性能基准测试。
            \item 产出一个能在内部解决特定问题的命令行原型。
        \end{enumerate}
\end{itemize}

\subsection{第二阶段:产品化与API开发}
\begin{itemize}
    \item \textbf{目标:} 将原型转化为一个稳定、可靠、可供外部调用的软件引擎。
    \item \textbf{团队配置:} 增加软件工程师 (C++ \& Python), QA工程师。
    \item \textbf{关键任务:}
        \begin{enumerate}
            \item 算法优化:提升求解速度、数值稳定性和内存效率。
            \item API设计:开发稳定、可扩展的C语言API,作为所有上层SDK的基础。
            \item 构建Python SDK,并实现许可证验证逻辑。
            \item 建立持续集成(CI)和自动化测试流程。
        \end{enumerate}
\end{itemize}

\subsection{第三阶段:商业化与生态系统建设}
\begin{itemize}
    \item \textbf{目标:} 构建完整的商业化基础设施,并推向市场。
    \item \textbf{团队配置:} 增加DevOps工程师, Web前端/后端工程师, 技术文档工程师。
    \item \textbf{关键任务:}
        \begin{enumerate}
            \item 搭建并部署许可证服务器和客户门户网站。
            \item 开发跨平台的安装包 (`.msi`, `.pkg`, `.deb/.rpm`)。
            \item 撰写并发布完善的官方文档和用户手册。
            \item 启动Beta测试计划,收集早期用户反馈。
        \end{enumerate}
\end{itemize}

\vspace{1cm}
\hrule
\vspace{1cm}
\begin{center}
    \textbf{总结:} 本报告详细阐述了我们求解器软件的技术架构、设计方向和潜在的自研路径。无论是基于现有架构进行优化,还是启动自研项目,都需要在算法研究、软件工程和商业化运营上进行协同投入。
\end{center}

\end{document}
  