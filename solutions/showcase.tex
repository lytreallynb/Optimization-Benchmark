\documentclass[10pt]{beamer}

% --- 主题与基础包 ---
\usetheme[progressbar=frametitle]{metropolis}
\usepackage{appendixnumberbeamer}
\usepackage{booktabs}
\usepackage[scale=2]{ccicons}
\usepackage{xspace}
\usepackage{amsmath} % 为数学公式添加支持
\usepackage{graphicx} % 在导言区引入

% --- 中文支持 ---
\usepackage{ctex}

% \usepackage{fontspec} % 字体设置
% \setmainfont{Fira Sans} % 设置主字体为 Fira Sans
% \setsansfont{Fira Sans} % 设置无衬线字体为 Fira Sans
% \setmonofont{Fira Code} % 设置等宽字体为 Fira


% --- 标题与作者信息 ---
\title{ASU优化基准平台}
\subtitle{理解与参与策略}
\date{} % 留空以不显示日期
\author{COPT战略分析团队}
\institute{内部战略研讨}

\begin{document}

% --- 标题页 ---
\maketitle

% --- 目录页 ---
\begin{frame}{目录}
  \setbeamertemplate{section in toc}[sections numbered]
  \tableofcontents
\end{frame}

% --- 第一部分:平台介绍 ---
\section{平台介绍}

\begin{frame}{平台概述}
  \textbf{什么是ASU优化基准?}
  \begin{itemize}
    \item \textbf{维护方}: 由亚利桑那州立大学(ASU)的Hans Mittelmann教授精心维护。
    \item \textbf{核心功能}: 为多种优化问题类型的求解器提供一个全面、权威的基准测试平台。
    \item \textbf{涵盖范围}: 覆盖LP、MILP、SDP、SOCP、NLP、QP及其他专业优化问题。
    \item \textbf{评估方法}: 使用\textbf{移位几何平均数(Shifted Geometric Mean)}进行严谨的性能比较。
    \item \textbf{资源提供}: 提供所有测试的日志文件和详细的性能分析报告,确保透明度和可追溯性。
  \end{itemize}
\end{frame}

\begin{frame}{可分析的基准类别}
  \textbf{我们可以分析的问题类型:}
  \begin{columns}[T,onlytextwidth]
    \column{0.5\textwidth}
      \begin{itemize}
        \item \textbf{线性规划 (LP)}
        \begin{itemize}
          \item LPfeas (可行性问题)
          \item LPopt (最优化问题)
          \item 大型网络LP问题
        \end{itemize}
        \item \textbf{混合整数规划 (MILP)}
        \begin{itemize}
          \item MIPLIB2017等标准基准
          \item 病态案例
          \item 不可行性检测
        \end{itemize}
        \item \textbf{半定规划 (SDP)}
        \begin{itemize}
          \item 稀疏SDP问题
          \item 不可行SDP问题
        \end{itemize}
      \end{itemize}
    \column{0.5\textwidth}
      \begin{itemize}
        \item \textbf{二阶锥规划 (SOCP)}
        \begin{itemize}
          \item 大型SOCP问题
        \end{itemize}
        \item \textbf{非线性规划 (NLP)}
        \begin{itemize}
          \item AMPL-NLP基准
        \end{itemize}
        \item \textbf{二次规划 (QP)}
        \begin{itemize}
          \item 凸/非凸QP
          \item 二进制/连续变量变体
        \end{itemize}
      \end{itemize}
  \end{columns}
\end{frame}

% --- 第二部分:机遇与战略 ---
\section{机遇与战略}

\begin{frame}{当前格局与机遇}
  \textbf{市场定位与发展机遇}
  \begin{block}{当前格局}
    在2018至2024年间,多家主流商业求解器(如 \textbf{CPLEX, Gurobi, XPRESS})已相继退出该基准测试。
  \end{block}
  
  \begin{alertblock}{我们的机遇}
    \begin{itemize}
        \item \textbf{开放空间}: 为新求解器创造了参与竞争、获得全球可见度的绝佳机会。
        \item \textbf{已有验证}: 根据Mittelmann基准数据,\textbf{COPT} 已展现出顶级性能。
        \item \textbf{活跃社区}: 通过\textbf{NEOS平台},已有约60个求解器参与,形成活跃的学术和研究社区。
    \end{itemize}
  \end{alertblock}
\end{frame}

\begin{frame}[allowframebreaks]{战略收益}
  \textbf{为什么参与至关重要?}
  \begin{itemize}
    \item \textbf{提升可见度 (Visibility)}
      \begin{itemize}
        \item 在全球优化社区获得权威认可与品牌曝光。
      \end{itemize}
    \item \textbf{性能验证 (Validation)}
      \begin{itemize}
        \item 通过独立第三方平台,客观验证求解器性能与稳定性。
      \end{itemize}
    \item \textbf{扩大研究影响 (Research Impact)}
      \begin{itemize}
        \item 被全球顶尖大学及研究机构在学术研究中参考使用。
      \end{itemize}
    \item \textbf{驱动持续优化 (Benchmarking)}
      \begin{itemize}
        \item 持续跟踪与对手的性能对比,为产品迭代提供数据支持。
      \end{itemize}
    \item \textbf{融入前沿社区 (Community)}
      \begin{itemize}
        \item 接触尖端测试问题,与全球研究者建立合作。
      \end{itemize}
  \end{itemize}
\end{frame}

% --- 第三部分:执行计划 ---
\section{执行计划}

\begin{frame}{参与流程}
  \textbf{如何提交我们的求解器?}
  \begin{columns}[T,onlytextwidth]
    \column{0.5\textwidth}
      \begin{block}{选项一:NEOS服务器集成}
        \begin{itemize}
          \item \textbf{联系方式}: \texttt{support@neos-server.org}
          \item \textbf{流程}: 提供求解器描述和文档,与NEOS平台进行技术集成。
          \item \textbf{优势}: 获得更广泛的用户可访问性和社区曝光度。
        \end{itemize}
      \end{block}
    \column{0.5\textwidth}
      \begin{block}{选项二:ASU直接托管}
        \begin{itemize}
          \item \textbf{联系方式}: Hans Mittelmann 教授 (\texttt{mittelmann@asu.edu})
          \item \textbf{流程}: 直接联系Mittelmann教授,在ASU服务器上托管求解器。
          \item \textbf{优势}: 更直接、更快速地参与基准测试。
        \end{itemize}
      \end{block}
  \end{columns}
  
  \begin{exampleblock}{基本要求}
    \begin{itemize}
      \item \textbf{求解器文档}: 清晰、完整的技术和使用文档。
      \item \textbf{性能测试能力}: 确保求解器能够稳定运行测试案例。
      \item \textbf{维护承诺}: 积极响应并进行维护(Mittelmann教授每周花费数小时更新)。
    \end{itemize}
  \end{exampleblock}
\end{frame}

\begin{frame}[allowframebreaks]{技术实施计划}
  \textbf{我们的方法与时间表}
  \begin{itemize}
    \item \textbf{第一阶段:分析与提取}
      \begin{itemize}
        \item 任务:提取并分析现有基准案例的数学模型与数据结构。
      \end{itemize}
    \item \textbf{第二阶段:开发处理管道}
      \begin{itemize}
        \item 任务:开发强大的Python处理管道,处理多种输入格式。
      \end{itemize}
    \item \textbf{第三阶段:提交与集成}
      \begin{itemize}
        \item 任务:提交我们的求解器,完成与NEOS或ASU平台的集成。
      \end{itemize}
    \item \textbf{第四阶段:监控与迭代}
      \begin{itemize}
        \item 任务:监控性能排名,分析结果,根据反馈迭代改进。
      \end{itemize}
  \end{itemize}
  \begin{alertblock}{预计时间表}
    我们在预期时间内完成完整的技术集成和初步的性能展示。
  \end{alertblock}
\end{frame}

\begin{frame}{下一步行动}
  \textbf{即刻启动,明确目标}
  \begin{columns}[T,onlytextwidth]
    \column{0.5\textwidth}
      \begin{block}{立即行动 (Immediate Actions)}
        \begin{enumerate}
          \item \textbf{模型提取}: 从Mittelmann网站提取基准测试案例的数学模型。
          \item \textbf{脚本开发}: 启动灵活的Python脚本开发工作,处理不同模型格式。
          \item \textbf{建立联系}: 与Hans Mittelmann教授建立初步联系,表达参与意向。
          \item \textbf{文档准备}: 整理并打包求解器的技术文档、安装指南和性能说明。
        \end{enumerate}
      \end{block}
    \column{0.5\textwidth}
      \begin{exampleblock}{成功指标 (Success Metrics)}
        \begin{itemize}
          \item \textbf{排名目标}: 在核心基准类别中获得有竞争力的性能排名。
          \item \textbf{集成成功}: 成功将求解器与NEOS平台集成并稳定运行。
          \item \textbf{持续优化}: 建立定期性能监控和优化的内部流程。
        \end{itemize}
      \end{exampleblock}
  \end{columns}
\end{frame}

% --- 第四部分:COPT求解器使用展示 (新增) ---
\section{COPT 求解器使用展示}

\begin{frame}[allowframebreaks]{产品设计理念:强大的“引擎”}
  \textbf{COPT本质上是一个优化求解器引擎,而非一个带图形界面的软件。}

  \begin{columns}[T,onlytextwidth]
    \column{0.5\textwidth}
      \begin{alertblock}{它是什么?}
        \begin{itemize}
          \item 一个高性能的数学计算核心。
          \item 通过编程接口(API)被调用。
          \item 主要面向开发者和数据科学家。
        \end{itemize}
      \end{alertblock}
    \column{0.5\textwidth}
      \begin{block}{它不是什么?}
        \begin{itemize}
          \item 一个像Excel那样的点击式应用。
          \item 不需要用户手动“操作界面”。
          \item 它的“界面”就是代码本身。
        \end{itemize}
      \end{block}
  \end{columns}

  \begin{center}
    \vspace{0.5cm}
    可以把它理解为汽车的\textbf{发动机},而不是整台汽车。开发者围绕这个“引擎”构建完整的应用程序。
  \end{center}
\end{frame}

\begin{frame}{核心工作流程}
  \textbf{从问题到答案,只需四步:}
  \begin{enumerate}[<+-| alert@+>]
    \item \textbf{引入库}: 在Python环境中,引入COPT功能库。
    \item \textbf{建立模型}: 使用API,将业务问题中的变量、约束、目标翻译成代码。
    \item \textbf{启动求解}: 调用简单命令,启动求解器引擎进行计算。
    \item \textbf{获取结果}: 从返回结果中,提取最优决策方案指导业务。
  \end{enumerate}
  \vspace{1cm}
  \begin{center}
  \large
  这个流程将复杂的运筹学理论,封装在简单易用的编程接口背后。
  \end{center}
\end{frame}

\begin{frame}[fragile,allowframebreaks]{API调用示例:一个简单的线性规划问题}
  \textbf{数学模型:}
  \begin{equation*}
    \begin{array}{ll@{}ll}
      \text{max}  & x + 2y \\
      \text{s.t.} & -x + y \leq 1 \\
                  & x + y \leq 2 \\
                  & x, y \geq 0
    \end{array}
  \end{equation*}

  \textbf{对应的Python代码:}

{\footnotesize
\begin{verbatim}
import coptpy as cp

# 1. 创建模型
mdl = cp.Envr().createModel("lp_example")

# 2. 创建变量
x = mdl.addVar(name="x")
y = mdl.addVar(name="y")

# 3. 添加约束
mdl.addConstr(-x + y <= 1)
mdl.addConstr(x + y <= 2)

# 4. 定义目标函数
mdl.setObjective(x + 2*y, cp.COPT.MAXIMIZE)

# 5. 求解模型
mdl.solve()

# 6. 打印结果
print("Optimal value: {}".format(mdl.objval))
print("x={}, y={}".format(x.x, y.x))
\end{verbatim}
}
\end{frame}

\begin{frame}[allowframebreaks]{API的价值:连接数学理论与商业实践}
  \textbf{API是翻译器,也是连接器,将抽象的数学模型转化为可执行的商业逻辑。}
  \begin{block}{如何帮助导出数学模型?}
    API提供了一套与数学符号高度对应的语言。研究员脑中的公式,可以直接“翻译”成代码。
    \begin{itemize}
      \item \textbf{数学公式}: $\sum_{i=1}^{n} c_i x_i \leq B$
      \item \textbf{对应代码}: \\ \texttt{mdl.addConstr(cp.quicksum(c[i]*x[i] for i in range(n)) <= B)}
    \end{itemize}
  \end{block}
  \begin{alertblock}{API的商业用途是什么?}
    \begin{itemize}
      \item \textbf{自动化决策}: 嵌入企业ERP、WMS等系统,实现供应链、排产、定价等自动优化。
      \item \textbf{定制化应用}: 根据独特业务场景,开发专属优化工具,而非使用通用软件。
      \item \textbf{“What-if”分析}: 快速调整模型参数,模拟不同市场环境下的最优策略,支持决策。
    \end{itemize}
  \end{alertblock}
\end{frame}

% --- 新增幻灯片 ---
\begin{frame}[allowframebreaks]{COPT 性能优势与展望}
  \begin{columns}[T,onlytextwidth]
    \column{0.48\textwidth}
      \begin{alertblock}{核心优势 (做得好)}
        \begin{itemize}
          \item \textbf{线性规划 (LP) \& 混合整数规划 (MIP)}: 根据Mittelmann榜单,COPT在核心领域性能达到世界顶尖水平。
          \item \textbf{二阶锥/二次/半定规划}: 在凸二次规划、二阶锥规划和半定规划领域,性能位居世界前列。
          \item \textbf{技术创新}: 发布支持GPU加速的一阶算法求解器,满足超大规模问题求解需求。
        \end{itemize}
      \end{alertblock}
    \column{0.48\textwidth}
      \begin{block}{持续发展方向 (做得更好)}
        \begin{itemize}
          \item \textbf{通用非线性规划 (NLP)}: 持续提升对非凸、非线性模型处理能力。
          \item \textbf{生态系统与社区}: 加强用户社区、第三方工具与文档建设,构建强大生态系统。
          \item \textbf{前沿算法探索}: 投入新算法研究,解决专门化和结构特殊问题,提高针对性和效率。
        \end{itemize}
      \end{block}
  \end{columns}
\end{frame}



% --- 结尾 ---
{\setbeamercolor{palette primary}{fg=black, bg=yellow}
\begin{frame}[standout]
  Q \& A
\end{frame}
}

\end{document}
